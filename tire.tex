% Created 2019-12-02 Mon 19:32
% Intended LaTeX compiler: pdflatex
\documentclass[11pt]{article}
\usepackage[utf8]{inputenc}
\usepackage[T1]{fontenc}
\usepackage{graphicx}
\usepackage{grffile}
\usepackage{longtable}
\usepackage{wrapfig}
\usepackage{rotating}
\usepackage[normalem]{ulem}
\usepackage{amsmath}
\usepackage{textcomp}
\usepackage{amssymb}
\usepackage{capt-of}
\usepackage{hyperref}
\usepackage{minted}
\usepackage{tikz} \usetikzlibrary{positioning, shapes.symbols, calc} \usepackage{svg}
\author{Vishakh Pradeep Kumar}
\date{\today}
\title{Tire Models}
\hypersetup{
 pdfauthor={Vishakh Pradeep Kumar},
 pdftitle={Tire Models},
 pdfkeywords={},
 pdfsubject={},
 pdfcreator={Emacs 26.3 (Org mode 9.2.6)}, 
 pdflang={English}}
\begin{document}

\maketitle
\tableofcontents

So your car might be the fanciest in the world but it still depends on four relatively small tyre contact patches.

We're gonna be ignoring the tread pattern for now because that is \textbf{way} too complicated. Damn it \sout{Jim} Life, I'm a \sout{doctor} plane guy not a car guy.

To make matters more interesting, these tyre models are really this mix between proper theoretical models and empirical models - your model might be well-used and simple but always take care to state your assumptions

\section{Stalking}
\label{sec:org1915446}

Formula Student Competition run by IMechE and takes place at SilverStone.


\section{Tyre Axis Frames \& Geometry}
\label{sec:org63424fa}
\subsection{SAE J2047}
\label{sec:org3632628}

So this tire frame is kind of a classic that a lot of textbooks use. While I could use a newer one (like ISO 8855), I'm gonna stick to the one I have since it's really similar but has switched signs. 

Latest reference of SAE J2047 is available at \url{https://www.sae.org/standards/content/j2047\_201911/}

Right off the bat, they tell you it's got a bunch of previous versions. We're using the latest one 

X axis is the intersection of the wheel plane and the road plane with the positive direction taken for the wheel moving forward.
Z axis is perpendicular to the road plane with a positive direction assumed to be acting downwards.
Y axis is in the road plane and its direction is dictated by the use of a right-handed orthogonal axis frame
The angles \(\alpha\) \& \(\gamma\) represent the slip angle and camber angle respectively.

\begin{figure}[!ht]
 \centering
 \includesvg[width=0.5\columnwidth, svgpath = ./]{saej2047}
\end{figure}
\end{document}
